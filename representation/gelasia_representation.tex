\documentclass[a4paper,11pt]{article}
\usepackage[T1]{fontenc}
\usepackage[utf8]{inputenc}
\usepackage{lmodern}
\usepackage[english]{babel}
\usepackage{amsmath}
\usepackage{graphicx}
\usepackage[labelformat=empty]{caption}
\usepackage[section]{placeins}
\usepackage{textcomp}
\usepackage{listings}

\lstdefinestyle{preformated}{
  language=C,
  stepnumber=1,
  numbersep=7pt,
  tabsize=2,
  showspaces=false,
  showstringspaces=false
}
\lstset{basicstyle=\small,style=preformated}

\title{Gelasia binary number representation.}
\author{Francisco Casas B.}

\begin{document}

\maketitle

\begin{abstract}
The present document shows a way to represent integer numbers of any size on a binary digits array, saving at the same time both it's value and it's size, this in order to store or send a group of well delimited numbers on a compact and scalable way, ensuring the small numbers to use less space. This can also be used when seting up standars, avoiding issues related to indexation (like, the overflow of a fixed space to countain addresses, or a very expensive memory reserve for a large number range, that's not totally used on most of the cases).
\end{abstract}

\tableofcontents

\section{Deduction}
\subsection{Previous analysis}
Normally, the amount of numbers that's possible to represent on a given space of $n$ bits is $2^n$. If we consider only positive integers (because starting from $0$ makes the deduction harder) the range of representable values is $[1,2^n]$ then, on a rough approach, if it's necessary to represent a number and delimitate it, we would take the number on it's binary form (after substracting $1$) and count the possition of the most significant 1 as the size of it, it's size has to be somehow stored, but that will be threated later, for now, the numbers will be represented this way:
\begin{center} \begin{lstlisting}
	 1 = [size:1] 0
	 2 = [size:1] 1
	 3 = [size:2] 10
	 4 = [size:2] 11
	 5 = [size:3] 100
	 6 = [size:3] 101
	 7 = [size:3] 110
	 8 = [size:3] 111
	 9 = [size:4] 1000
	10 = [size:4] 1001
	11 = [size:4] 1010
	12 = [size:4] 1011
	13 = [size:4] 1100
	14 = [size:4] 1101
	15 = [size:4] 1110
	16 = [size:4] 1111
	17 = [size:5] 10000
	      ...
\end{lstlisting} \end{center}

Then, if both the number and it's size are known, it's logical to think that if a number is big enough to requiere it's size, it's possible to dischard that the number has values that can be represented with smaller sizes:
\begin{center} \begin{lstlisting}
	 1 = [size:1] 0
	 2 = [size:1] 1
	 3 = [size:2] 00
	 4 = [size:2] 01
	 5 = [size:2] 10
	 6 = [size:2] 11
	 7 = [size:3] 000
	 8 = [size:3] 001
	 9 = [size:3] 010
	10 = [size:3] 011
	11 = [size:3] 100
	12 = [size:3] 101
	13 = [size:3] 110
	14 = [size:3] 111
	15 = [size:4] 0000
	16 = [size:4] 0001
	17 = [size:4] 0010
	      ...
\end{lstlisting} \end{center}
Now every possible value is being used, so, to get a number $N$ given it's size $S$ and this value $X$ (that will be called extra from now), the next formula will be used.
\begin{align*}
	N&= X +1 + \sum_{1\leq i < S}{2^i} \\
	&= X+1+ \frac{2^{S}-2}{2-1} \\
	&= X+2^{S}-1
\end{align*}
Now, since the size $S$ of a number, is also a number, it can be represented the same way, but, because $S>0$, it's better to store $S-1$. We can do so, until the size$-1$ of a size is $0$, The result is as follows:
\begin{center} \begin{lstlisting}
	 1 = [size:1] 0
	 2 = [size:1] 1
	 3 = [size:2] 00 = [size:1] 0 00
	 4 = [size:2] 01 = [size:1] 0 01
	 5 = [size:2] 10 = [size:1] 0 10
	 6 = [size:2] 11 = [size:1] 0 11
	 7 = [size:3] 000 = [size:1] 1 000
	 8 = [size:3] 001 = [size:1] 1 001
	 9 = [size:3] 010 = [size:1] 1 010
	10 = [size:3] 011 = [size:1] 1 011
	11 = [size:3] 100 = [size:1] 1 100
	12 = [size:3] 101 = [size:1] 1 101
	13 = [size:3] 110 = [size:1] 1 110
	14 = [size:3] 111 = [size:1] 1 111
	15 = [size:4] 0000 = [size:2] 00 0000 = [size:1] 0 00 0000
	16 = [size:4] 0001 = [size:2] 00 0001 = [size:1] 0 00 0001
	17 = [size:4] 0010 = [size:2] 00 0010 = [size:1] 0 00 0010
	      ...	
\end{lstlisting} \end{center}
The only information that's then needed to recover a number besides the secuence of digits on the right is how many times the size-1 representation algorithm was applied (this will be called recursivity), since this number grows on a very slow rate it can be represented on the most primitive way that allows an infinite size, this is a series of $1$ ended by a $0$, where the amount of $1$'s is the number of times that the algorithm had to be applied:
\begin{center} \begin{lstlisting}
           1 = 0 0 
           2 = 0 1
           3 = 10 0 00
           4 = 10 0 01 
           5 = 10 0 10 
           6 = 10 0 11 
           7 = 10 1 000 
           8 = 10 1 001 
           9 = 10 1 010 
          10 = 10 1 011 
          11 = 10 1 100 
          12 = 10 1 101 
          13 = 10 1 110 
          14 = 10 1 111 
          15 = 110 0 00 0000 
          16 = 110 0 00 0001 
          17 = 110 0 00 0010 
          18 = 110 0 00 0011 
	      		...	
\end{lstlisting} \end{center}
Not only the number it's stored on these bits, also it's size, that allows many numbers to be added contiguously on a bit array without need of a delimitator of additional data.
\subsection{Unsigned gelasia representation}
Now, with the last results, it's possible to substract $1$ to the values in order to have a representation for the $0$. This results on the following final representation:
\begin{center} \begin{lstlisting}
           0 = 0 0 
           1 = 0 1
           2 = 10 0 00
           3 = 10 0 01 
           4 = 10 0 10 
           5 = 10 0 11 
           6 = 10 1 000 
           7 = 10 1 001 
           8 = 10 1 010 
           9 = 10 1 011 
          10 = 10 1 100 
          11 = 10 1 101 
          12 = 10 1 110 
          13 = 10 1 111 
          14 = 110 0 00 0000
          25 = 110 0 00 1011 
         125 = 110 0 10 111111 
         625 = 110 1 001 001110011 
        3125 = 110 1 011 10000110111 
       15625 = 110 1 101 1110100001011 
       78125 = 1110 0 00 0000 0011000100101111 
      390625 = 1110 0 00 0010 011111010111100011 
     1953125 = 1110 0 00 0100 11011100110101100111 
     9765625 = 1110 0 00 0111 00101010000001011111011 
    48828125 = 1110 0 00 1001 0111010010000111011011111 
   244140625 = 1110 0 00 1011 110100011010100101001010011 
  1220703125 = 1110 0 00 1110 001000110000100111001110010111 
  6103515625 = 1110 0 01 00000 01101011110011000100000111101011 
 30517578125 = 1110 0 01 00010 1100011010111111010100100110001111 
152587890625 = 1110 0 01 00101 0001110000110111100100110111111000011
	             ...	
\end{lstlisting} \end{center}
Where a number is determinated by the next formula, the calculation of the size $S$ will be shown later:
\begin{align*}
	N &= X+2^{S}-2
\end{align*}
\subsection{Signed gelasia representation}
It's also possible to add a sign bit when necessary, not substracting the $1$ above mentioned to the absolute value of the negative numbers, in order to get only one representation of the value $0$. This sign bit will be $1$ for negative numbers and $0$ otherwise:
\begin{center} \begin{lstlisting}
	      ...
	-15 = 1 110 0 00 0000 
	-14 = 1 10 1 111 
	-13 = 1 10 1 110 
	-12 = 1 10 1 101 
	-11 = 1 10 1 100 
	-10 = 1 10 1 011 
	 -9 = 1 10 1 010 
	 -8 = 1 10 1 001 
	 -7 = 1 10 1 000 
	 -6 = 1 10 0 11 
	 -5 = 1 10 0 10 
	 -4 = 1 10 0 01 
	 -3 = 1 10 0 00
	 -2 = 1 0 1
	 -1 = 1 0 0 
	  0 = 0 0 0 
	  1 = 0 0 1
	  2 = 0 10 0 00
	  3 = 0 10 0 01 
	  4 = 0 10 0 10 
	  5 = 0 10 0 11 
	  6 = 0 10 1 000 
	  7 = 0 10 1 001 
	  8 = 0 10 1 010 
	  9 = 0 10 1 011 
	 10 = 0 10 1 100 
	 11 = 0 10 1 101 
	 12 = 0 10 1 110 
	 13 = 0 10 1 111 
	 14 = 0 110 0 00 0000 
	        ...	
\end{lstlisting} \end{center}
Then, if B is the sign bit of a number, it's value would be:
\begin{align*}
	N&= (X+2^{S}-2)(1-2B)-B
\end{align*}

\section{Calculation of a number's size}
For a positive number $N$ (before substracting $1$) the size $S$ is the position (from right to left, starting from $0$) of the most significant $1$ of the binary representation of the value $N+1$, this also applies to a negative number with it's absolute value. And is deducted from the equation $N = X+2^{S}-1$.
\\ Once the value $S$ is known, the following amounts of bits should be added to $S$ to complete the total size.
\begin{center}
	\begin{tabular}{ r || c| c || c}
		S & recursivity & size indication & total \\
		\hline 
		1 & 1 & 0 & S+1 \\
		2-3 & 2 & 1 & S+3 \\
		4-7 & 3 & 3 & S+6 \\
		8-15 & 3 & 4 & S+7 \\
		16-31 & 4 & 7 & S+11 \\
		32-63 & 4 & 8 & S+12 \\
		64-127 & 4 & 9 & S+13 \\
		128-255 & 4 & 10 & S+14 \\
		256-511 & 4 & 12 & S+15 \\
		512-1023 & 4 & 13 & S+16 \\
		1024-2047 & 4 & 14 & S+17 \\
		2048-4095 & 4 & 15 & S+18 \\
		4096-8191 & 4 & 16 & S+19 \\
		8192-16383 & 4 & 17 & S+20 \\
		16384-32767 & 4 & 18 & S+21 \\
		32768-65535 & 4 & 19 & S+22 \\	  
		65536-131071 & 5 & 23 & S+28 \\
		131072-262143 & 5 & 24 & S+29
	\end{tabular}
\end{center} 
Then, if an unsigned number is smaller than $2^{21}-1=2097151$ (before substracting $1$) it's granteed to use the same or less of the space that's possible to allocate on 32 bits.
	If the representation is signed, $1$ extra bit should be added.
\section{Possible variantions}
It's possible to set the minimal recursivity to $0$ instead of $1$, this will give a natural representation for the number $0$ that only requires one bit, but will imply for all the other values to have an aditional $1$ on it's recursivity header.
\begin{center} \begin{lstlisting}
           0 = 0
           1 = 10 0 
           2 = 10 1
           3 = 110 0 00
           4 = 110 0 01 
           5 = 110 0 10 
           6 = 110 0 11 
           7 = 110 1 000 
           8 = 110 1 001 
           9 = 110 1 010 
          10 = 110 1 011 
          11 = 110 1 100 
          12 = 110 1 101 
          13 = 110 1 110 
          14 = 110 1 111 
          15 = 1110 0 00 0000 
          16 = 1110 0 00 0001 
          17 = 1110 0 00 0010 
          18 = 1110 0 00 0011 
	      		...	
\end{lstlisting} \end{center}
Also it's possible to add a base value to the size of all the numbers, for example $3$:
\begin{center} \begin{lstlisting}
           0 = 0 000
           1 = 0 001
           2 = 0 010
           3 = 0 011
           4 = 0 100
           5 = 0 101
           6 = 0 110
           7 = 0 111
           8 = 10 0 00 0000
           9 = 10 0 00 0001
          24 = 10 0 01 00000
           	   ...	
\end{lstlisting} \end{center}
This modifications, among others, like applying an offset, can be made to ensure that the most frequent numbers use less space.
\end{document}
